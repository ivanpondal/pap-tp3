\section{Ejercicio 3}

Peso asignado: X.

\subsection{Problema}

En este problema nos dan una situación en la que cada docente tiene un sueldo, que es un entero positivo, determinado por la suma de
todos los cargos inferiores y el mismo y a su vez por todas las antigüedades menores y la misma. De esa manera, nos piden encontrar
la diferencia de sueldo entre el cargo actual y un cargo futuro.

Como entrada recibimos los enteros $C$ (cantidad de cargos), $A$ (niveles de antigüedad), $Q$ (cantidad de queries), una matriz $M$
de tamaño $C \times A$ y las Q queries con formato $<c_1, a_1, c_2, a_2>$.

A partir de esto se define el sueldo de un docente con cargo $c$ y antigüedad $a$ como:

\begin{equation*}
\begin{split}
    S_{c,a} & = \sum_{i=1}^{c}{\sum_{j=1}^{a}{M_{i,j}}} \\
    		& = M_{c,a} + \sum_{i=1}^{c-1}{\sum_{j=1}^{a}{M_{i,j}}} + 
    		\sum_{i=1}^{c}{\sum_{j=1}^{a-1}{M_{i,j}}} - 
    		\sum_{i=1}^{c-1}{\sum_{j=1}^{a-1}{M_{i,j}}} \\
    		& = M_{c,a} + S_{c-1,a} + S_{c,a-1} - S_{c-1,a-1}
\end{split}
\end{equation*}

Mediante esta información, vamos a calcular el resultado de cada query $<c_1, a_1, c_2, a_2>$ como:

\begin{equation*}
\begin{split}
    Q_{c_1,a_1,c_2,a_2} & = \sum_{i=c_1+1}^{c_2}{\sum_{j=a_1+1}^{a_2}{M_{i,j}}} \\
    					& = \sum_{i=c_1+1}^{c_2}{(\sum_{j=1}^{a_2}{M_{i,j}} - \sum_{j=1}^{a_1}{M_{i,j}})} \\
    					& = \sum_{i=c_1+1}^{c_2}{\sum_{j=1}^{a_2}{M_{i,j}}} - \sum_{i=c_1+1}^{c_2}{\sum_{j=1}^{a_1}{M_{i,j}}} \\
    					& = \sum_{i=1}^{c_2}{\sum_{j=1}^{a_2}{M_{i,j}}} - \sum_{i=1}^{c_1}{\sum_{j=1}^{a_2}{M_{i,j}}} -
    					\sum_{i=1}^{c_2}{\sum_{j=1}^{a_1}{M_{i,j}}} + \sum_{i=1}^{c_1}{\sum_{j=1}^{a_1}{M_{i,j}}} \\
    					& = S_{c_2,a_2} - S_{c_1,a_2} - S_{c_2,a_1} + S_{c_1,a_1}
\end{split}
\end{equation*}

\subsection{Algoritmo}

\subsubsection*{Explicación y correctitud}

El algoritmo consiste en dos partes: encontrar $S_{c,a}$ para cada valor posible y $Q_{c_1,a_1,c_2,a_2}$
para cada query.

Para hacer esto vamos a modelar $S$ como una matriz que completaremos al principio del 
algoritmo con tal de tenerla calculada para cada query. Cada query $Q_{c_1,a_1,c_2,a_2}$ a su vez, sera calculada
a partir de los valores de $S$.

Como se puede ver en las ecuaciones anteriores, calcular cada $S_{c,a}$ se puede hacer fácilmente a partir de los valores adyacentes
en la matriz. Es esencialmente un problema de programación dinámica.

Y las ecuaciones mismas garantizan la correctitud del algoritmo.

$Observaci\acute{o}n: $ la idea parte de implementar una Tabla Aditiva Bidimencional $S$ para $M_{i,j}$ y calcular las queries a partir
de los `cuadrados de suma' que esta genera.

\subsubsection*{Pseudocodigo}

\begin{algorithm}[H]
    \caption{CalcularDiferenciaSueldos}
    \Input{Enteros $C$, $A$, $Q$, Matriz de enteros $M$ de tamaño $C \times A$, Conjunto de queries $queries$ de tamaño $Q$}
    Matriz de enteros $S \gets$ Matriz de enteros de tamaño $C \times A$ rellena de 0 \;
    $S_{0,0} \gets M_{0,0}$ \;
    \For {Entero $i \gets 1$ ; $i < C$; $i \gets i + 1$} {
    	$S_{i, 0} \gets S_{i-1, 0} + M_{i, 0}$
    }
    \For {Entero $i \gets 1$ ; $i < A$; $i \gets i + 1$} {
    	$S_{0, i} \gets S_{0, i-1} + M_{0, i}$
    }
    \For {Entero $i \gets 1$ ; $i < C$; $i \gets i + 1$} {
        \For {Entero $j \gets 1$ ; $j < A$; $j \gets j + 1$} {
            $S_{i,j} = M_{i,j} + S_{i-1,j} + S_{i,j-1} - S_{i-1,j-1}$
        }
    }
    \For {cada query $<c_1, a_1, c_2, a_2>$ en $queries$} {
    	Entero $solucion \gets S_{c_2,a_2} - S_{c_1,a_2} - S_{c_2,a_1} + S_{c_1,a_1}$ \;
        Imprimir solucion
    }
\end{algorithm}

\subsection{Complejidad}

El algoritmo:
\begin{itemize}
	\item Crea una matriz de tamaño $C \times A$, $O(C \times A)$.
	\item Hace una asignación, $O(1)$.
	\item Realiza tres ciclos for con asignaciones, sumas, restas y comparaciones para calcular la matriz $S$. El primero 
	recorre $C-1$ valores de $i$ (trabajo $O(C)$). El segundo recorre $A-1$ valores de $i$ (trabajo $O(A)$). El 
	tercero recorre $C-1$ valores de $i$ y dentro, $A-1$ valores de $j$ (trabajo $O(C \times N)$). En total los
	ciclos tienen una complejidad de $O(C + A + C \times A) = O(C \times A)$.
	\item Hace Un ultimo ciclo For para resolver las queries. Para cada una de las $Q$ queries calcula la solución en
	base a sumas y restas de valores ya definidos. Como son operaciones constantes, la complejidad total del ciclo es $O(Q)$.
\end{itemize}

Por lo tanto, la complejidad total es de $O(C \times A + 1 + C \times A + Q) = O(C \times A + Q)$.

\subsection{Casos de prueba}

\subsubsection*{Caso 1}

Probamos que funcione el algoritmo para el caso provisto por la cátedra.

\begin{center}
    \begin{tabular}{| l | l |}
    \hline
    Entrada del ejemplo 1 & Salida del ejemplo 1  \\ \hline
    5 5 4 & 24 \\
	1 2 3 4 5 & 0 \\
	2 3 4 5 6 & 96 \\
	3 4 5 6 7 & 20 \\
	4 5 6 7 8 & \\
	5 6 7 8 9 & \\
	2 2 4 4  & \\
	3 3 3 5  & \\
	1 1 5 5  & \\
	1 2 3 4  & \\
	\hline
    \end{tabular}
\end{center}

\subsubsection*{Caso 2}

Probamos que funcione el algoritmo para un caso mediano donde sea fácil de chequear si la respuesta es correcta.

\begin{center}
    \begin{tabular}{| l | l |}
    \hline
    Entrada del ejemplo 2 & Salida del ejemplo 2  \\ \hline
	10 10 5 & 0 \\
	1 1 1 1 1 1 1 1 1 1 & 0 \\
	1 1 1 1 1 1 1 1 1 1 & 0 \\
	1 1 1 1 1 1 1 1 1 1 & 0 \\
	1 1 1 1 1 1 1 1 1 1 & 81 \\
	1 1 1 1 1 1 1 1 1 1 & \\
	1 1 1 1 1 1 1 1 1 1 & \\
	1 1 1 1 1 1 1 1 1 1 & \\
	1 1 1 1 1 1 1 1 1 1 & \\
	1 1 1 1 1 1 1 1 1 1 & \\
	1 1 1 1 1 1 1 1 1 1 & \\
	1 1 1 5 & \\
	1 1 1 10 & \\
	1 1 5 1 & \\
	1 1 10 1 & \\
	1 1 10 10 & \\
	\hline
    \end{tabular}
\end{center}

\subsubsection*{Caso 3}

Probamos que funcione el algoritmo y corra en un tiempo lógico para un input del mayor tamaño posible.

\begin{center}
    \begin{tabular}{| l | l |}
    \hline
    Entrada del ejemplo 3 & Salida del ejemplo 3  \\ \hline
	1000 100 100000 & 0 \\
	1 2 3 4 ... 100 & 0 \\
	2 3 4 5 ... 101 & 0 \\
	3 4 5 6 ... 102 & 0 \\
	... & 0 \\
	... & 0 \\
	1000 1001 1002 1003 ... 1099 & 0 \\
	1 1 1 1 & 0 \\
	1 1 1 2 & 0 \\
	1 1 1 3 & 0 \\
	... & \\
	... & \\
	1 1 1 100 & 0 \\
	1 1 2 1 & 0 \\
	... & \\
	... & \\
	1 1 1000 100 & 54494451 \\
	\hline
    \end{tabular}
\end{center}
