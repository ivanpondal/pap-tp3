\section{Ejercicio 4}

Peso asignado: X.

\subsection{Introducción}

El objetivo de este ejercicio era brindar un algoritmo que tomando como entrada
$D$ días con un número entero positivo denotando su diversión, y $R$ intervalos
sobre los días, retornara la suma de los dos días de mayor diversión para cada
rango. Además, la complejidad temporal del mismo debía estar acotada por
\ord($D + R \times \log{D}$).

\subsection{Solución propuesta}

El problema planteado es uno de consultas sobre intervalos donde para cada uno
de estos rangos se pide poder responder de forma rápida cuál es la suma de los
dos eventos más entretenidos.

Para esto se tienen varias estructuras que habiendo definido un operador binario
asociativo permiten aplicarlo eficientemente a distintos rangos del mismo
arreglo. Sin embargo, la operación que interesa utilizar para el ejercicio en
cuestión no es binaria.

La misma se puede definir de la siguiente manera:

\begin{equation*}
	\emph{sumaDosMásDivertidos}(A, i, j) = \max(A\left[i, j\right)) +
	\max(A\left[i, j\right) - \left\{\max(A\left[i, j\right))\right\})
\end{equation*}

Como se puede observar no es más que sumarle al máximo del intervalo, el máximo
sin ese elemento. Entonces uno podría reescribir la operación de forma de llegar
a la siguiente formulación:

\begin{align*}
	\emph{sumaDosMásDivertidos} & (A, i, j) = \\
	& \max(A\left[i, j\right)) + \max( \\
	& \max(A\left[i, pos(A, \max(A\left[i, j\right))\right)), \\
	& \max(A\left[pos(A, \max(A\left[i, j\right))) + 1, j\right)) \\
	& )
\end{align*}

Donde \emph{pos} devuelve la posición en el arreglo de la primera aparición del
elemento que se le pasa por parámetro.

Por lo tanto es posible realizar la operación pedida únicamente consultando por
el máximo, que es un operador binario y asociativo, lo cual nos permite
utilizar las estructuras estudiadas.

A la hora de decidir qué estructura implementar es necesario tener en
consideración la complejidad temporal pedida de \ord($D + R \times \log{D}$).

Se analizarán las siguientes estructuras:

\begin{itemize}
	\item{Tabla Adtitiva}
	\item{Sparse Table}
	\item{Segment Tree}
\end{itemize}

\subsubsection*{Tabla Aditiva}

Utilizar una \emph{Tabla Aditiva} tiene el beneficio de que su inicialización
tiene un costo lineal y las consultas se responden en tiempo constante. El
problema con ella es que para consultar por intervalos es necesario que la
operación posea una inversa. Como no existe la inversa del máximo, esta opción
queda descartada.

\subsubsection*{Sparse Table}

Con \emph{Sparse Table} no es necesario que la operación tenga una inversa y las
consultas se pueden responder en tiempo logarítmico, sin embargo, la
inicialización tiene un costo \ord($N \times \log N$) donde $N$ es el tamaño del
arreglo original. Puesto que para el problema el tamaño del arreglo es $D$ y un
costo \ord($D \times \log D$) no está acotado por \ord($R \times \log{D}$) ya
que $R$ no tiene por qué estar acotado por $D$, no es posible utilizar
\emph{Sparse Table} para resolver el problema.

\subsubsection*{Segment Tree}

Queda \emph{Segment Tree}, que no requiere tener la inversa de la operación,
tiene un costo de inicialización lineal y puede responder consultas en tiempo
logarítmico. Es así entonces que se decidió implementar la misma para poder
calcular el máximo de forma rápida en los intervalos necesarios para computar la
suma de los dos días más entretenidos.

~

A continuación se presenta el pseudocódigo del algoritmo que resuelve el problema:

\begin{algorithm}[H]
	\caption{Suma dos días de mayor diversión en intervalo}
	\Input{Arreglo \emph{días} de longitud $D$ con enteros positivos denotando
	la diversión para cada día y arreglo \emph{consultas} con pares $[i, j)$.}
	\Output{Para cada consulta devuelve el valor de la suma de los dos días de
	mayor diversión comprendidos en tal rango.}
	\emph{segmentTree} $\gets$ crear \emph{Segment Tree} con los valores del
	arreglo \emph{días} \;
	\ForEach{\emph{consulta} \textbf{en} \emph{consultas}} {
		\emph{diversiónMáxima} $\gets$ \emph{segmentTree}.max(\emph{consulta}.i, \emph{consulta}.j) \;
		\emph{segundaDiversiónMáxima} $\gets$
		$\max$(\emph{segmentTree}.max(\emph{consulta}.i,
		\emph{diversiónMáxima}.pos),
		\emph{segmentTree}.max(\emph{diversiónMáxima}.pos + 1, \emph{consulta}.j))
		\;
		\emph{imprimir}(\emph{diversiónMáxima}.valor +
		\emph{segundaDiversiónMáxima}.valor) \;
	}
\end{algorithm}

\subsection{Complejidad teórica}

Se demostrará que el código desarrollado efectivamente tiene una complejidad
temporal acotada por \ord($D + R \times \log{D}$). Para ello se separará en
\ord($D$) y \ord($R \times \log{D}$).

La inicialización del \emph{Segment Tree} toma el arreglo de tamaño $D$ y si no
es potencia de dos hace que lo sea. Luego completa la estructura aplicando la
operación seleccionada a cada nivel del árbol.

Lo primero, llevar el arreglo a uno de tamaño potencia de dos está acotado por
\ord($D$) ya que a lo sumo debe extenderlo al doble. Completar el resto del
árbol consiste en ir de las hojas hasta la raíz, donde en cada paso el valor del
padre corresponde a la operación aplicada a sus hijos, esto también está acotado
por \ord($D$) ya que teniendo la representación del árbol en forma de arreglo es
cuestión de recorrer linealmente del último hasta el primer elemento. Es
necesario aclarar que además del valor del máximo, en cada nodo se almacena la
posición del mismo. Por lo tanto la inicialización queda acotada por \ord($D$).

Queda ver que resolver las consultas tiene un costo de \ord($R \times \log{D}$).
En total se deben responder $R$ consultas, por lo tanto lo que se demostrará es
que las mismas se pueden responder en tiempo \ord($\log{D}$).

Dado que se utiliza un $Segment Tree$, se sabe que el costo de encontrar el máximo
para un intervalo en particular tiene costo \ord($\log{D}$). Así ya se tiene el
primer valor necesario para obtener la suma buscada. El valor del segundo día de
mayor entretenimiento se obtiene analizando los dos intervalos resultantes al
quitar el máximo encontrado en el punto anterior. Para ello se busca el máximo
en cada uno de estos intervalos, nuevamente \ord($\log{D}$). Por último se toma
el máximo entre ambos, \ord(1). Finalmente se suman ambos valores, también con
un costo constante.

Por lo tanto, encontrar la suma de los dos días de mayor diversión para un
intervalo en particular tiene un costo \ord($3 \times \log{D}$) =
\ord($\log{D}$).

Sumando el costo de inicialización junto al de responder las $R$ consultas se
concluye que la complejidad temporal del algoritmo es \ord($D + R \times
\log{D}$).

\subsection{Casos de prueba}

Para evaluar el código desarrollado además de ver que diera los resultados
correctos en las entradas provistas por el enunciado se realizaron las
siguientes pruebas:

\begin{itemize}
	\item{Entradas de tamaño potencia de dos}
	\item{Entradas de tamaño distinto a una potencia de dos}
	\item{Entradas con $D = 10^5$ y $R = 10^5$}
\end{itemize}

Las entradas de tamaño igual y distintas a una potencia de dos se utilizaron
para comprobar que la inicialización del \emph{Segment Tree} funcionara
correctamente. Para ambos casos se probaron los siguientes intervalos:

\begin{itemize}
	\item{Intervalo abarcando la totalidad del arreglo}
	\item{Intervalo abarcando el medio del arreglo}
	\item{Intervalo abarcando el comienzo del arreglo}
	\item{Intervalo abarcando el final del arreglo}
\end{itemize}

Las entradas con $D = 10^5$ y $R = 10^5$ tenían como propósito corroborar que el
algoritmo pudiera manejar tamaños grandes sin problemas. Para ello se generó un
arreglo de tamaño $10^5$ con días de igual diversión y $10^5 - 1$ consultas por
intervalos de tamaño dos. Así únicamente era necesario asegurar que la suma
fuera igual a el doble de la diversión asignada a cada día.
