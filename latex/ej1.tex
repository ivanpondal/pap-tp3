\section{Ejercicio 1}

Peso asignado: 8.

\subsection{Introducción}

En esta sección detallamos la resolución del ejercicio 1 del trabajo práctico. Para eso, primero extraemos del enunciado una especificación del problema que plantea y luego identificamos un problema equivalente a él, desmotrando esa equivalencia. En lo siguiente nos concentramos en proponer un algoritmo que resuelva el segundo problema.

\subsubsection{Notación}

Dados dos strings $A$ y $B$, notamos $AB$ a su concatenación. En general, los strings son nombrados con letras mayúsculas. Cuando nos referimos a strings con letras minúsculas, estos siempre son strings de un solo caracter. Un tal string también se denominará ``caracter'' de forma indistinta.

\subsection{Problema}

El enunciado del problema plantea un escenario en el que se desea la capacidad de determinar si un string es \textit{substring} de otro con una determinada restricción de eficiencia.

Formalmente, Dados los strings $T$ y $S$, $T$ será substring de $S$ si existe un prefijo $S_1$ de $S$ y un sufijo $S_2$ de $S$ tales que $S = S_1 T S_2$. Notemos que:
\begin{itemize}
\item Para que el problema tenga sentido, la longitud de $T$, notada $|T|$, debe ser menor o igual a la de $S$, i.e, es una precondición del problema que $|T| \leq |S|$.
\item Tanto $S_1$ como $S_2$ pueden ser el \textit{string vacío}, que consideraremos de longitud $0$, en cuyo caso $T$ sería prefijo o sufijo de $S$ respectivamente.
\end{itemize}

En el caso particular de este enunciado, el mismo restringe que la longitud de ambos strings de entrada del problema sean no-vacíos.

En este contexto, se pide diseñar un algoritmo que, dados dos strings $T$ y $S$ tales que $0 < |T| \leq |S|$, informe si $T$ es substring de $S$ en tiempo lineal en la longitud de $S$. Ocasionalmente llamaremos \textit{patrón} a $T$ y \textit{texto} a $S$.

\subsection{Un problema equivalente}

\subsubsection{Definiciones}

Un substring $B$ de $R$ con $|R| > 0$ se dice \textit{borde de R} si es prefijo y sufijo de $R$. Diremos que el borde $B$ de $R$ es \textit{propio} si $B \neq R$. En particular, definiremos al borde propio $B$ de $R$ como \textit{no-cruzado} si existe un substring $M$ de $R$ tal que $R = BMB$. Un borde se denominará \textit{cruzado} si no es \textit{no-cruzado}. Notemos que tanto el substring vacío como el string entero $R$ son trivialmente bordes de $R$; sin embargo, $R$ no es borde no-cruzado ni propio de sí mismo.

Ejemplos:
\begin{itemize}
\setlength\itemsep{0em}
\item ``'' es borde no-cruzado y propio de cualquier string no-vacío.
\item ``bab'' es borde cruzado y propio de ``babab''.
\item ``abc'' es borde cruzado y no es propio de ``abc".
\item ``abc'' es borde no-cruzado de ``abcxyzabc'' ya que ``abcxyzabc'' = ``abc''``xyz''``abc'', y es propio.
\item ``abc'' es borde no-cruzado de ``abcabc'' ya que ``abcabc'' = ``abc''``''``abc'', y es propio.
\end{itemize}

\subsubsection{Propiedad:} Dados dos strings $T$ y $S$ tales que $0 < |T| \leq |S|$:
$T$ es substring de $S$ $\iff$ algún prefijo de $TS$ tiene un borde no-cruzado de longitud $|T|$.

\subsubsection{Demostración:}

$\Longrightarrow) \ $$T$ es substring de $S$ \spaciousimply existen $S_1$ y $S_2$ tales que $S = S_1 T S_2$ \spaciousimply $TS = T S_1 T S_2$ \spaciousimply $T S_1 T$ es un prefijo de $TS$ que tiene un borde no-cruzado de longitud $T$.

$\Longleftarrow) \ $Sea $P = B P_1 B$ un prefijo de $TS$ con borde no-cruzado $B$ tal que $|B| = |T|$, de modo que $TS = P S_1 = B P_1 B S_1$. Como tanto $B$ y $T$ son prefijos de $TS$ del mismo tamaño, entonces $B = T$. Por lo tanto, $TS = T P_1 T S_1$, lo cual equivale a $S = P_1 T S_1$ quitando $T$ a izquierda de ambos lados. Entonces, $T$ es substring de $S$.

\subsubsection{Conclusión}

En conclusión, dada una entrada válida $T, S$ del problema planteado en la sección anterior, determinar si $T$ es substring de $S$ equivale a determinar si algún prefijo de $TS$ tiene un borde no-cruzado de tamaño $T$. El algoritmo que presentamos se piensa como solución al segundo problema.

\subsection{Algoritmo}

Denominamos \textit{esSubstring} al algoritmo que resuelve el problema. El algoritmo consiste en dos etapas. Dado el patrón $T$ y el texto $S$, en la primera etapa se reúne información acerca de las longitudes de los bordes de todos los prefijos de $TS$. Luego, en una segunda instancia, se procesa la información recopilada en la primera para decidir si $T$ es substring de $S$ usando la propiedad enunciada en la sección anterior.

\subsubsection{Lema de extensión de bordes}

\textbf{Definición.} En general, dado un string no vacío $S$, definimos la \textit{sucesión de bordes propios de S} como $B_1, B_2, \dots, B_r, r > 0$ donde $B_1$ es el mayor borde propio de $S$, $B_2$ es el mayor borde propio de $B_1$, y en general el string $B_j$ es el mayor borde propio de $B_{j-1}$, $1 < j \leq r$.

El elemento $B_r$ en todos los casos será el string vacío, siendo que siempre tomamos bordes propios cuyas longitudes necesariamente van decreciendo, y eventualmente llegamos a ``'' a quien es imposible tomar bordes propios. Observemos que, por lo tanto, para cualquier string no vacío esta sucesión siempre tiene al string vacío como mínimo.

\medskip

\textbf{Lema.} Dado el string no-vacío $S$ y su sucesión de bordes propios $A_S = B_1, \dots, B_r$, el mayor borde propio del string $Sc$ para un caracter $c$ es

\begin{itemize}
\setlength\itemsep{0em}
\item O bien $Bc$, siendo $B$ el string más largo de $A_S$ al cual le sigue el caracter $c$ en $S$, si tal $B$ existe.
	Esto vale pues los bordes de $Sc$ no son más que los bordes de $S$ a los que les sigue $c$.
	Como todos los bordes propios de $S$ están en $A_S$, $B$ solo puede ser el string más largo de $A_S$;
\item O bien el string vacío, si tal $B$ no existe (es decir, a todos los bordes propios de $S$ le sigue un caracter distinto de $c$).
\end{itemize}

\subsubsection{Lema de bordes anidados}

Sea $S$ un string no vacío con borde $B$ y un string $B'$ tal que $|B'| < |B|$. Entonces
\[
	B' \text{ es borde de } B \iff B' \text{ es borde de } S.
\]

\medskip

$\Longrightarrow) \ $ Si $B'$ es borde de $B$, entonces es prefijo de $B$ que a su vez es prefijo de $S$, con lo cual $B'$ es prefijo de $S$. A su vez, $B'$ es sufijo de $B$ que es sufijo de $S$, por lo que $B'$ también es sufijo de $S$. Por definición, al ser $B'$ tanto prefijo como sufijo de $S$, es borde de $S$.

$\Longleftarrow) \ $ Como $B'$ es prefijo de $S$ de longitud menor que $|B|$, entonces es prefijo de $B$ ya que $B$ es prefijo de $S$. A su vez, $B'$ es sufijo de $S$ de longitud menor que $|B|$, por lo que es sufijo de $B$ ya que $B$ es sufijo de $S$. Como $B'$ es, por ende, tanto prefijo como sufijo de $B$, es borde de $B$.

\subsubsection{Primera etapa}

Damos el algoritmo \textit{BordesPropiosDePrefijosConLongitud} que, dado un string no-vacío de entrada $S$ y un natural $m$, devuelve un diccionario que a cada posible longitud de prefijo de $S$ (los $|S|+1$ naturales entre 0 y $|S|$) asocia un booleano que indica si tiene un borde propio de longitud $m$. Dado que todos los prefijos de un string particular pueden identificarse sin ambigüedad por su longitud, este diccionario contiene toda la información necesaria para el procesamiento de la segunda etapa. Notaremos $P_l$ al prefijo de longitud $l$ en el contexto de un string particular. Observemos que nos alcanza con tratar las longitudes de los bordes \textit{propios} de los prefijos ya que nos interesará en particular evaluar bordes no-cruzados, y estos siempre son propios en strings no-vacíos.

\bigskip

\begin{algorithm}[H]
	\caption{BordesPropiosDePrefijosConLongitud}
	\Input{ string $S$ no vacío y entero $m > 0$ }
	\Output{ diccionario $T$ de naturales en booleanos }

	$T \gets$ inicializar con claves de 0 a $|S|$ \\
	$F \gets$ inicializar con claves de 0 a $|S|$ y todos los valores en \textit{falso} \\
	$T [0] \gets -1$ \\
	$T [1] \gets 0$ \\
	$k \gets 0$ \\
	
    \For {$i$ de 2 a $|S|$} {
	\While {$k \neq -1 \wedge S[i-1] \neq S[k]$} {
		$k \gets T[k]$
    	}
	$k \gets k+1$ \\
	$T[i] \gets k$ \\
	$F[i] \gets \: k = m \; \vee \; F[k]$
}

	\Return{F}	
\end{algorithm}

\bigskip

En términos generales, el diccionario $T$ irá guardando las longitudes de los mayores bordes propios de los prefijos de $S$ a medida que avanza, sirviendo así para ir extendiendo los bordes en base al lema de extensión de bordes. A su vez, para cada prefijo al que se calcula la longitud de su mayor borde propio, en $F$ se guarda si ese prefijo tiene un borde de longitud igual a $m$ utilizando la información en $T$. Observemos que podemos suponer que $S$ siempre tiene tamaño mayor o igual a 2, ya que la entrada de esta etapa del algoritmo siempre es la concatenación del patrón y el texto que conforman la entrada del problema, que son ambos strings no-vacíos por precondición. 

\medskip

El prefijo vacío no interesa en sí a efectos de este problema ya que no tiene bordes propios, por lo que a $T$ en 0 se asocia el valor conveniente ad-hoc de $-1$ que facilita la escritura del código. A su vez, $T$ en 1 siempre será 0 ya que el único borde propio de un string de longitud 1 es el vacío. Notemos que $T$ siempre tendrá al menos tres claves siendo que su tamaño es $|S|+1$ y $|S| \geq 2$ por lo explicado arriba, con lo cual estas operaciones están siempre bien definidas.

\medskip

En el comienzo de la iteración $i$ del ciclo externo, el algoritmo tiene:
\begin{itemize}
\setlength\itemsep{0em}
\item En $k$ la longitud del mayor borde propio de $P_{i-1}$ calculado en una etapa anterior. De esta forma, $k$ también es la posición en $S$ inmediatamente posterior a la del último caracter de ese borde, en la que se encuentra el caracter que se agregaría a dicho borde si fuera a extenderse por uno.
\item En $T[j]$ para todo $0 \leq j < i$ la longitud del mayor borde propio $P_j$.
\item En $F[j]$ para todo $0 \leq j < i$ un booleano que dice si $P_j$ tiene un borde de longitud exactamente $m$
\end{itemize}

En esa $i$-ésima iteración, los objetivos del algoritmo son:
\begin{itemize}
\setlength\itemsep{0em}
\item Calcular la longitud del mayor borde propio de $P_i$, que puede escribirse como $P_i = P_{i-1}c$ extendiendo a $P_{i-1}$ con el caracter correspondiente, a partir de lo conocido sobre $P_{i-1}$.
\item Determinar si $P_i$ tiene un borde de longitud $m$.
\end{itemize}

Para el primer objetivo, el algoritmo sigue el lema de extensión de bordes exhibido arriba: busca el string de mayor longitud en la sucesión de bordes propios de $P_{i-1}$ al que le sigue el caracter $c$, y si no lo encuentra se queda con el borde vacío. Esta búsqueda, realizada en el ciclo interno, esencialmente recorre la sucesión de bordes propios de $P_{i-1}$: si esta sucesión es $A_{P_{i-1}} = B_1, \dots, B_r$, la variable $k$ va tomando los valores de las longitudes de los elementos de $A_{P_{i-1}}$ uno por uno en orden decreciente a medida que no se encuentra un borde al que le sigue el caracter $c$ en $S$; observemos que, de manera análoga a lo mencionado arriba respecto de $k$, estos valores también representan la posición inmediatamente siguiente en $S$ de cada string de la sucesión.

En el ciclo interno, entonces, ni bien se encuentra un borde $B_0$ extensible con $c$, se sabe por el lema que $B_{0}c$ es el mayor borde propio de $P_i = P_{i-1}c$, se corta el ciclo interno y se asocia a $P_i$ en $T$ la longitud $|B_{0}c| = k+1$; si no encuentra ninguno, se sabe por el lema que el mayor borde propio de $P_i$ es el vacío y se asocia a $P_i$ en $T$ la longitud 0.

Para ver que la forma de iterar la sucesión de bordes propios $A = B_1, \dots, B_r$ en el ciclo interno es correcta, analicemos la iteración de ese ciclo en la que $k = |B_j|$. El string $B_j$ es el borde de un prefijo de $S$, y por lo tanto es un prefijo de $S$ de longitud $k$, es decir, $B_j = P_k$. Al ejecutarse el cuerpo del ciclo, $k$ toma el valor asociado a $k$ en $T$, que corresponde a información acerca del mayor borde propio del prefijo de tamaño $k$ de $S$, $P_k$. O sea, la variable $k$ que tenía el valor $|B_j|$ pasa a ser la longitud del (y la posición en $S$ inmediatamente posterior al) mayor borde propio de $P_k = B_j$, que es justamente $B_{j-1}$ por definición de la sucesión de bordes propios. Esta operación además es válida ya que el ciclo corta ni bien $k$ toma el valor ad-hoc $-1$ correspondiente al último elemento de la sucesión de  bordes propios, el string vacío.

\medskip

Para el segundo objetivo, una vez conocida la longitud del mayor borde propio de $P_i$ guardada en la variable $k$ en la línea 10, el algoritmo debe determinar si $P_i$ tiene algún borde propio, no necesariamente el mayor, de longitud exactamente $m$. Si en la iteración $i$ resulta que $k$ es justo $m$, esto es trivialmente cierto y el algoritmo asocia \textit{verdadero} a $F$ en $i$, valor que no se modificará. Si no, revisa si el borde recién encontrado de $P_i$, que es el prefijo $P_k$, tiene un borde de longitud $m$. Esto lo hace consultando el diccionario $F$ en $k$, quien donde se encuentra información por invariante. Por el lema de bordes anidados, cualquiera sea ese borde de longitud $m$ de $P_k$, será también borde de $P_i$. Más aún, si $P_k$ no tiene un borde de longitud $m$, por el mismo lema se determina que tampoco lo tiene $P_i$, ya que si lo tuviera, este sería un borde de longitud $m$ de $P_k$ y $F$ en $k$ sería verdadero.
 
\subsubsection{Algoritmo final}

Sabiendo cómo obtener la información necesaria sobre los bordes propios de los prefijos de un string con \textit{BordesPropiosDePrefijosConLongitud}, podemos escribir el algoritmo \textit{esSubstring} que, usando el algoritmo anterior como primera etapa, busca determinar si algún prefijo de $TS$ tiene un borde no-cruzado de longitud igual a $|T|$.

\bigskip

\begin{algorithm}[H]
	\caption{esSubstring}
	\Input{ patrón $T$ y texto $S$ }
	\Output{ booleano que indica si $T$ es substring de $S$ }

	$M \gets \mathit{BordesPropiosDePrefijosConLongitud}(TS, |T|)$
	
	\For {$i$ de $2|T|$ a $|TS|$} {
		\If {$M[i]$} {
			\Return{\textit{verdadero}}
		}
	}

	\Return{\textit{falso}}	
\end{algorithm}

\bigskip

El algoritmo corre \textit{BordesPropiosDePrefijosConLongitud} sobre $TS$ y guarda su resultado en $M$, que mapea cada longitud de prefiojs de $TS$ ($0$ a $|TS|$) a un booleano que indican si ese prefijo posee un borde de longitud |T|. Su objetivo es, como se dijo arriba, determinar específicamente si alguno de esos prefijos de $TS$ tiene un borde no-cruzado de longitud $|T|$, lo cual equivaldría a que $T$ es substring de $S$ por la propiedad enunciada anteriormente.

Esta búsqueda, realizada en el ciclo de la línea 2, comienza en el prefijo de tamaño $2|T|$ de $TS$. Para explicar esto, veamos que cualquier borde de tamaño $T$ de un prefijo de tamaño menor a $2|T|$ es necesariamente un borde cruzado.

Sea $B$ es un borde de $P_m$ con $m < 2|T|$ tal que $|B| = |T|$. Supongamos que existe un string $X$ tal que $P_m = BXB$. Entonces, debe cumplirse $|P_m| = m = 2|B|+|X| = 2|T| + |X| \geq 2|T|$, lo cual es absurdo habiendo supuesto $m < 2|T|$.

Del mismo modo, cualquier borde de tamaño $|T|$ que se encuentre en un prefijo de tamaño mayor o igual a $2|T|$ naturalmente será no-cruzado. Además, vale observar las claves de $M$ ordenadas crecientemente siempre tendrán, en particular, valores mayores o iguales a $2|T|$, ya que $|M| = |TS|+1 = |T|+|S|+1 \geq 2|T| + 1$ dado que $|S| \geq |T|$ por precondición del problema.

En conclusión, el algoritmo revisa uno a uno la información contenida en $M$ acerca de los prefijos que pueden llegar a tener bordes no-cruzados de longitud $|T|$, que son los de tamaños $2|T| \dots |TS|$. Ni bien encuentra un prefijo tal que lo cumple, lo cual equivale a haber determinado que $T$ es substring de $S$ por la propiedad mencionada, el algoritmo devuelve un resultado afirmativo. Si no encuentra ninguno, lo cual por la misma propiedad equivale a que $T$ \textit{no} es substring de $S$, devuelve un resultado negativo.
 
\subsection{Complejidad}

\subsubsection{Detalles de implementación}

Se supone una estructura de diccionario que permite la inicialización de una instancia de tamaño $n$ en tiempo $O(n)$ y la lectura y escritura de definiciones en $O(1)$. Dado que las claves de nuestro diccionario son $0 \dots m$ para un $m > 0$, el diccionario puede ser implementado mediante un arreglo de $m+1$ posiciones cuyos índices representan las claves. Además, se supone la capacidad de concatenar un par de strings $A$ y $B$ en tiempo $O(|A|+|B|) = O(|AB|)$ y de acceder a sus caracteres en $O(1)$.

\subsubsection{Complejidad de la primera etapa}

El algoritmo comienza definiendo los diccionarios $T$ y $F$ con tamaño $|S|+1$ en tiempo $O(|S|)$ y el entero $k$.

Luego le sigue un ciclo entre las líneas 6-13 (al que llamaremos A), que a su vez contiene otro ciclo
entre las líneas 7-9 (al que llamaremos B).

El ciclo A itera $|S|-1$ veces y dentro realiza asignaciones y comparaciones en orden $O(1)$, aparte del ciclo B. Es decir que
sacando el ciclo B, el ciclo A tiene complejidad $O(|S|)$. Lo que se quiere ver es que el ciclo B tiene
complejidad $O(|S|)$ en total, es decir, entre todas las iteraciones del ciclo A.

El ciclo A cumple lo siguiente:
\begin{itemize}
	\item $T[j] < j$ para todo $j$, esto vale antes de entrar al ciclo
	y veremos que la asignación de la línea 10 lo preserva.
	\item $k$ aumenta a lo sumo en 1 por cada iteración del ciclo A (mediante la línea 9), esto vale pues por la
	propiedad anterior, $k$ solo puede disminuir en el ciclo B.
	\item $k$ está acotado superiormente por $|S|-1$, ya que por la propiedad anterior, nunca
	aumenta más de 1 vez por iteración.
	\item $k < i$ en cada iteración del ciclo, esto vale antes de entrar al ciclo
	y como $i$ aumenta en 1 por cada iteración y $k$ a lo sumo aumenta en 1, se preserva.
	\item como $k < i$, la línea 10 preserva $T[i] < i$.
\end{itemize}

Analizando cuántas veces se puede ejecutar el ciclo B vemos que esto depende del valor de $k$.
Teniendo en cuenta los invariantes anteriores sabemos que $T[k] < k$, por lo cual en la línea 7,
$k$ decrece. Ahora bien, por cada vez que $k$ aumenta en 1, la línea 7 puede disminuir $k$ en al menos 1.

Por ejemplo supongamos que estamos en una iteración del ciclo A de tal forma que nos faltan procesar $n+t$
letras y $k == 0$. Después supongamos que las primeras $t$ iteraciones del ciclo A no entran en el ciclo B
(y así $k$ aumenta en $t$). Luego en la iteración siguiente de A se entra por el ciclo B hasta que
$k == -1$. Entonces como faltan procesar $n$ letras, $k$ será como máximo $n$ al terminar el ciclo A, 
ya que por cada ciclo aumenta a lo sumo en 1, pero a su vez el ciclo B de ahora en más solo podrá
decrecer $n$ veces.

Por lo tanto, por cada vez que $k$ aumenta en 1 en la línea 9, se podrá entrar 1 vez más en el ciclo B. Esto
quiere decir que la cantidad de iteraciones del ciclo B esta acotado por el máximo valor de $k$, que ya
vimos en las propiedades del ciclo A, esta acotado por $|S|-1$. Entonces el ciclo B se repite a lo sumo $|S|-1$
veces.

Teniendo en cuenta todo lo anterior, la complejidad total es:
$O($(inicializar el arreglo T) + (trabajo del ciclo A sin contar el ciclo B) + (complejidad total del ciclo B)$)
= O(|S| + |S| + |S|) = O(|S|)$.

\subsubsection{Complejidad total}

En \textit{esSubstring} se comienza concatenando $T$ y $S$ en tiempo $O(|TS|)$ invocando \textit{BordesPropiosDePrefijosConLongitud} con el string $TS$ y $|T|$ como parámetro. Esta invocación agrega un costo de $O(|TS|)$ como vimos en la sección anterior. Luego, el ciclo que sigue realiza $O(|TS|-2|T|) = O(|S|-|T|) = O(|S|)$ iteraciones (recordemos que por precondición $|T| \leq |S|$), donde en cada iteración se realizan $O(1)$ operaciones. En total, entonces, nuestro algoritmo tiene un tiempo de ejecución de $O(2|TS|) + O(S) = O(|TS|)$. A su vez, $O(|TS|) = O(|T|+|S|) = O(2|S|) = O(|S|)$ ya que $|T| \leq |S|$ por precondición, lo cual cumple la restricción requerida en el enunciado.

\subsection{Casos de prueba}

Utilizamos casos de prueba correspondientes a las siguientes familias:
\begin{itemize}
\item El patrón es substring propio del texto.
\item El patrón es prefijo del texto.
\item El patrón es sufijo del texto.
\item El patrón es igual al texto.
\item Casos variados de resultado negativo.
\item Casos en los que la sucesión de bordes propios de la concatenación del patrón y el texto no es trivial.
\item Casos con bordes cruzados del tamaño del patrón.
\item Casos en los que la concatenación del patrón y el texto tiene al menos un prefijo que posee tanto un borde de tamaño $|T|$ como de tamaños más grandes (para verificar que el algoritmo registre correctamente la existencia de bordes de tamaño $|T|$ a pesar de haber bordes de tamaño mayor).
\item Casos con la entrada del tamaño máximo sugerido por el enunciado para testear carga.
\end{itemize}


